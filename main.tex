%!TEX program = xelatex

\documentclass{ctexbeamer}

\usepackage[usepdf]{ustcbeamer}% 使用PDF形式的背景文件,编译速度快
% \usepackage[usetikz,ustcred]{ustcbeamer} % 使用tikz形式的背景文件,式样丰富

                          %%% ustcbeamer说明 %%%%
% 宏包默认选项"usepdf"使用pdf形式的背景文件(在子文件夹theme中),蓝色的主题且不可更改。

% 如若要使用其他颜色的主题,请使用选项"usetikz",则文档会默认加载 tikz 代码生成的背景
% "usetikz"搭配了另外四个选项 "ustcblue","ustcred","ustviolet","ustcblack"。
% "usetikz"会很大程度上损失编译速度,建议写文档的时候使用"usepdf",内容写完后再改成"usetikz",然后挑选自己喜爱的主题色选项。
%%%%%%%%%%%%%%%%%%%%%%%%%%%%%%%%%%%%%%%%%%%%%%%%%%%%%%%%%%%%%%%%%%%%%%


\title[底部简明标题]{
    自旋压缩的量子控制理论及应用
}
\author[底部演讲者]{报告人:XXX}
\institute[USTC]{
理论物理研究组\\
中国科学技术大学,近代物理系
}
\date{\today}
\begin{document}
%\section<⟨mode specification⟩>[⟨short section name⟩]{⟨section name⟩}
%小于等于六个标题为恰当的标题

%--------------------
%标题页
%--------------------
\maketitleframe
%--------------------
%目录页
%--------------------
%beamer 101
\begin{frame}%
	\frametitle{大纲}%
	\tableofcontents[hideallsubsections]%仅显示节
	%\tableofcontents%显示所节和子节
\end{frame}%
%--------------------
%节目录页
%--------------------
\AtBeginSection[]{
\setbeamertemplate{footline}[footlineoff]%取消页脚
  \begin{frame}%
    \frametitle{大纲}
	%\tableofcontents[currentsection,subsectionstyle=show/hide/hide]%高亮当前节,不显示子节
    \tableofcontents[currentsection,subsectionstyle=show/show/hide]%show,shaded,hide
  \end{frame}
\setbeamertemplate{footline}[footlineon]%添加页脚
}
%--------------------
%子节目录页
%--------------------
\AtBeginSubsection[]{
\setbeamertemplate{footline}[footlineoff]%取消页脚
  \begin{frame}%
    \frametitle{大纲}
	%\tableofcontents[currentsection,subsectionstyle=show/hide/hide]%高亮当前节,不显示子节
    \tableofcontents[currentsection,subsectionstyle=show/shaded/hide]%show,shaded,hide
  \end{frame}
\setbeamertemplate{footline}[footlineon]%添加页脚
}

\section{研究背景}
\begin{frame}
  \frametitle{研究背景}
  研究背景:
  \begin{itemize}
    \item 一
    \item 二
    \item 三
  \end{itemize}

\end{frame}


\section{理论模型}
\subsection{模型1}
\begin{frame}
  \frametitle{理论模型1}
  \begin{equation*}
    S\left(\rho\right)=-Tr\rho\ln\rho
  \end{equation*}
  \pause
  由此得到……

\end{frame}
\subsection{模型2}
\begin{frame}
  \frametitle{理论模型2}
  \begin{equation*}
    S\left(\rho\right)=-Tr\rho\ln\rho
  \end{equation*}
  \pause
  由此得到……

\end{frame}


\section{研究方法}

\begin{frame}
  \frametitle{研究方法}
  \begin{block}{方法一}
    \begin{itemize}
      \item abc
      \item def
    \end{itemize}
  \end{block}
  \pause
  \begin{block}{方法二}
    \begin{itemize}
      \item abc
      \item def
    \end{itemize}
  \end{block}
\end{frame}


\section{总结展望}

\begin{frame}
  \frametitle{总结展望}
  \begin{columns}
    \begin{column}{0.50\textwidth}
      \begin{figure}
        \includegraphics[width=0.8\textwidth]{figures/ustc_logo.pdf}
        \caption{标题}
      \end{figure}
    \end{column}
    \begin{column}{0.50\textwidth}
      \begin{block}{结论}
        \begin{itemize}
          \item 结论 1
          \item 结论 2
          \item 结论 3
        \end{itemize}
      \end{block}
    \end{column}
  \end{columns}
\end{frame}

\begin{frame}
  \frametitle{致谢}
  \centerline{\Large 谢谢!}
\end{frame}

\end{document}
