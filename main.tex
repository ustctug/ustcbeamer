%!TEX program = xelatex

\documentclass[12pt]{ctexbeamer}
\usepackage[ustcblue]{ustcbeamer}%可通过更改选项来更改主题颜色

%%%模板说明%%%%
%模板ustc_beamer中定义了五个选项供选择:ustcblue,ustcred,black,violet,blue;分别对应了五种主题颜色。
%建议使用ustcblue和ustcred,两者均为科大党委宣传部规定的校徽标准配色,参考http://lswhw.ustc.edu.cn/index.php/index/info/3370 。两个标准配色分别为:蓝cmyk(100,80,0,0)、红cmyk(0,100,100,0),在LaTex中使用需要除以100。
%本模板参考了https://github.com/thomasWeise/ustcSlides 的Slides,故而保留了Thomas Weise先生的原始配色(blue)。
%不怎么建议使用黑色(black),看上去像讣告。
%有其他的配色需求可在github上反馈。
%%%%%%%%%%%%%%%%%%%%%%%%%%%%%%%%%%%%%%%%%%%%%%%%%%%%%%%%%%%%%%%%%%%%%%

\title[底部简明标题]{
    量子力学熵形式的不确定关系
}
\author[底部演讲者]{报告人:XXX}
\institute[USTC]{
理论物理研究组\\
中国科学技术大学,近代物理系
}
\date{\today}

\begin{document}

\startPresentation{}

\AtBeginSection[]{
  \begin{frame}%
    \frametitle{大纲}
    \tableofcontents[currentsection]
  \end{frame}
}
\section{研究背景}
\begin{frame}
  \frametitle{研究背景}
  研究背景:
  \begin{itemize}
    \item 一
    \item 二
    \item 三
  \end{itemize}
\end{frame}


\section{理论模型}

\begin{frame}
  \frametitle{理论模型}
  \begin{equation*}
    S\left(\rho\right)=-Tr\rho\ln\rho
  \end{equation*}
  \pause
  由此得到……

\end{frame}


\section{研究方法}

\begin{frame}
  \frametitle{研究方法}
  \begin{block}{方法一}
    \begin{itemize}
      \item abc
      \item def
    \end{itemize}
  \end{block}
  \pause
  \begin{block}{方法二}
    \begin{itemize}
      \item abc
      \item def
    \end{itemize}
  \end{block}
\end{frame}


\section{总结展望}

\begin{frame}
  \frametitle{总结展望}
  \begin{columns}
    \begin{column}{0.50\textwidth}
      \begin{figure}
        \includegraphics[width=0.8\textwidth]{figures/ustc_logo.pdf}
        \caption{标题}
      \end{figure}
    \end{column}
    \begin{column}{0.50\textwidth}
      \begin{block}{结论}
        \begin{itemize}
          \item 结论 1
          \item 结论 2
          \item 结论 3
        \end{itemize}
      \end{block}
    \end{column}
  \end{columns}
\end{frame}


\begin{frame}
  \centerline{\Large 谢谢!}
\end{frame}


\end{document}
